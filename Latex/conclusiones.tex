 \chapter{Conclusiones}
Es innegable el continuo crecimiento de la Web 2.0 y lo que, entre otras cosas, esto conlleva: la expresión por parte de multitud de usuarios de la opinión de forma libre sobre cualquier aspecto de la sociedad. En un mundo donde prácticamente todo está automatizado e informatizado, resulta casi imprescindible tener en cuenta estas opiniones a la hora de la toma de decisiones por parte de cualquier corporación u organización.

El crecimiento del ámbito de estudio del NLP (\textit{Natural Language Processing}) y, por lo tanto, del análisis de sentimientos, nos lleva directamente al nacimiento de herramientas o modelos que puedan ayudar a realizar estas tareas de extraer conocimiento de textos y opiniones.

En este proyecto se han comparado 7 herramientas distintas para la extracción de polaridad y se ha probado a hacerlas funcionar de forma conjunta mediante el ensamblado. La principal conclusión es, sin duda, que el funcionamiento de cada herramienta depende mucho de cuál es el contexto de los comentarios, debido a que cada herramienta ha sido configurada o entrenada sobre distintos tipos de comentarios. El ejemplo más claro se ha visto con CoreNLP, que consigue puntuaciones buenas en \textit{datasets} que tratan de críticas sobre películas, contexto en el que ha sido entrenado y, además, obtiene resultados malos en el resto de conjuntos de datos.

Junto al contexto de los textos, se unen otras dos dificultades que son las faltas de ortografía u otros errores gramaticales y peculiaridades semánticas, como pueden ser la ironía y el sarcasmo. Ante la primera dificultad pocas herramientas pueden adaptarse adecuadamente y resulta complicado encontrar una solución buena que no sea un corrector ortográfico cuyo funcionamiento sea similar al que usamos en un teléfono móvil, pudiendo suponer esta solución una fase del preprocesamiento de los datos. La segunda dificultad es mucho más complicada, pues un humano también puede fallar a la hora de entender una ironía. A pesar de esto, hay herramientas como VADER que intentan dar solución a esta dificultad mediante funciones heurísticas.

Hemos observado también como en conjuntos de datos como vader\_nyt o sentistrength\_bbc, que tienen como fuente de comentarios los foros de discusión de los sitios webs de \textit{New York Times} y la \textit{BBC} respectivamente, obtienen muy malos resultados. Estos comentarios pueden depender unos de los anteriores y esto dificulta aun más el asignar una polaridad si la herramienta no tiene en cuenta el contexto.

En cuanto al ensamblado de las herramientas mediante la distribución de pesos se aprecia en los 2 modelos propuestos que no se alcanzan unos resultados del todo favorables pero, aun así, se mejora en algunos conjuntos de datos. Los malos resultados se deben a que existen herramientas que pueden introducir ruido en la predicción. Mediante algoritmos evolutivos pretendemos hacer que los pesos a las distintas herramientas sean asignados según la relevancia y la calidad de las herramientas en los distintos conjuntos de datos. De esta manera se consigue mejorar hasta en 11 conjuntos de datos los mejores resultados mostrados por las herramientas de forma individual y por los 2 ensamblados propuestos.

Ante los resultados hay que decir que el análisis de sentimientos es aun un área con mucho que explorar y que, seguramente, obtenga importantes resultados en los próximos años. Aparentemente el empleo de las herramientas para la extracción de polaridad aun está lejos de dar unos resultados que nos lleven a que podamos emplear estas herramientas como único método para la extracción de conocimiento desde textos. Sin embargo, resulta una fuente de información importante que puede usarse como información auxiliar a la de otras fuentes.

Desde mi experiencia en este proyecto afirmaría que emplear las herramientas como único método para el análisis de sentimientos en un proyecto real no sería lo más correcto. De hecho, herramientas como MeaningCloud seguramente a sabiendas de que el contexto es muy importante a la hora de predecir el sentimiento, ofrece la configuración personalizada del modelo pudiendo añadir un usuario diccionarios o sus propios conjuntos de datos de comentarios para entrenar modelos con estos. Una posible utilidad y que sería interesante estudiar es el empleo de las herramientas (de forma individual o ensamblada) como características adicionales a un problema de clasificación para la clasificación del sentimiento.

\subsubsection{Posibles trabajos futuros}
A partir de este proyecto se plantean posibles futuros trabajos. Podría ser interesante, por ejemplo, crear una herramienta que empleara algoritmos evolutivos para crear un modelo genérico (empleando un ensamblado de herramientas) para la predicción del sentimiento de un comentario.

Otros de los trabajos interesantes sería el investigar cómo funcionaría CoreNLP siendo entrenado con comentarios en un contexto diferente aprovechando que CoreNLP se trata de una herramienta de código abierto.

Como se ha comentado previamente puede resultar también interesante probar a usar estas polaridades devueltas por las herramientas (o por sus ensamblados) como características a un modelo de aprendizaje automático para la predicción de polaridades. Estos modelos de aprendizaje automático basados normalmente en la frecuencia y relevancia de términos o de \textit{n-gramas}, como el realizado en \cite{ana}, pueden no tener en cuenta aspectos como las negaciones o ironías que algunas herramientas sí son capaces de predecir en algunos casos. Por lo tanto, esta información puede ser útil a la hora de crear un modelo.