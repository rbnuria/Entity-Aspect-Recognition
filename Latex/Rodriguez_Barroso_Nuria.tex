%-------------------------- model.tex -------------------------%
%Exemple per a usar l'estil tfm.class per a la preparaci\'o
%de monografies de treballs final de m\`aster.
%
%
%----------------------------- FMETFM ----------------------------%
\documentclass[reqno,twoside, 12pt]{report}

\usepackage[spanish,english]{babel}
\usepackage[table,xcdraw]{xcolor}
\usepackage{pdfpages}
\usepackage[utf8]{inputenc}
\usepackage{amsmath}
\usepackage{hyperref}
\usepackage{mathtools}
\usepackage{hyperref}
\usepackage{graphicx}
\usepackage{algorithm}
\usepackage{algpseudocode}
\usepackage{parskip}
\usepackage{xcolor,colortbl}
\usepackage{fancyhdr}
\usepackage[export]{adjustbox}
\usepackage{rotating}
\usepackage{wrapfig}
\usepackage{epstopdf}
\usepackage{float}
\usepackage{eurosym} 
\usepackage{subcaption}
\usepackage{lscape}
\usepackage{lipsum}
\usepackage{amsfonts}
\usepackage{tikz}
\usetikzlibrary{matrix,chains,positioning,decorations.pathreplacing,arrows}

%Para introducir trozos de código
\usepackage{listings}
\lstloadlanguages{R}
\lstset{language=R,
	basicstyle=\small\ttfamily,
	stringstyle=\color{dkgreen},
	otherkeywords={0,1,2,3,4,5,6,7,8,9},
	morekeywords={TRUE,FALSE},
	deletekeywords={\_},
	keywordstyle=\color{blue},
	commentstyle=\color{DarkGreen},
}


\usepackage{color}
\definecolor{gray97}{gray}{.97}
\definecolor{mauve}{rgb}{0.58,0,0.82}
\definecolor{dkgreen}{rgb}{0,0.6,0}
\definecolor{greentable}{rgb}{0.176, 0.682, 0.396}

%hasta aquí
\usepackage{caption} 
\usepackage{pdflscape}
\usepackage{appendix}
\definecolor{blue}{rgb}{0.372, 0.847, 0.835}
\usepackage[table]{xcolor}
\newcommand{\specialcell}[2][c]{%
	\begin{tabular}[#1]{@{}l@{}}#2\end{tabular}}

\newtheorem{definition}{Definition}[chapter]

\pagestyle{fancy}
\fancyhf{}
\fancyhead[LE,RO]{}
\fancyhead[RE,LO]{\rightmark}
\fancyfoot[CE,CO]{\thepage}
\fancyfoot[LE,RO]{}

\renewcommand{\headrulewidth}{2pt}
\renewcommand{\footrulewidth}{1pt}

\newcolumntype{L}[1]{>{\raggedright\let\newline\\\arraybackslash\hspace{0pt}}m{#1}}
\newcolumntype{C}[1]{>{\centering\let\newline\\\arraybackslash\hspace{0pt}}m{#1}}
\newcolumntype{R}[1]{>{\raggedleft\let\newline\\\arraybackslash\hspace{0pt}}m{#1}}

\setcounter{tocdepth}{5}

%%-------------------------------------------------------------%
\begin{document}

% Seleccioneu l'idioma:
%\selectlanguage{catalan}
\selectlanguage{spanish}
%\selectlanguage{english}

%----------------------------------------------------------------
% Inici del document
%------------------------------------------------------------------
\begin{titlepage}
 
\newlength{\centeroffset}
\setlength{\centeroffset}{-0.5\oddsidemargin}
\addtolength{\centeroffset}{0.5\evensidemargin}
\thispagestyle{empty}

\noindent\hspace*{\centeroffset}\begin{minipage}{\textwidth}

\centering
\includegraphics[width=0.6\textwidth]{imagenes/logo_ugr.jpg}\\[1.4cm]

\textsc{ \Large TRABAJO FIN DE GRADO\\[0.2cm]}
\textsc{ DOBLE GRADO EN INGENIERÍA INFORMÁTICA Y MATEMÁTICAS}\\[1cm]
% Upper part of the page
% 
% Title
{\Huge\bfseries Análisis de sentimientos \\
}
\noindent\rule[-1ex]{\textwidth}{3pt}\\[3.5ex]
{\large\bfseries Aplicación de Deep Learning para la extrancción de características.}
\end{minipage}

\vspace{0.1cm}
\noindent\hspace*{\centeroffset}\begin{minipage}{\textwidth}
\centering

\textbf{Autor}\\ {Nuria Rodríguez Barroso}\\[2.5ex]
\textbf{Directores}\\
{Francisco Herrera Triguero\\
Mª Victoria Luzón García}\\[2cm]
\includegraphics[width=0.3\textwidth]{imagenes/etsiit_logo.png}\\[0.1cm]
\textsc{Escuela Técnica Superior de Ingenierías Informática y de Telecomunicación}\\
\textsc{---}\\
Granada, septiembre de 2018
\end{minipage}
%\addtolength{\textwidth}{\centeroffset}
%\vspace{\stretch{2}}
\end{titlepage}




%-----------------------------------------------------------------
% P�gina inicial
%-----------------------------------------------------------------
%\title{Sentiment Analysis For Touristic Attractions: \\
%A Case Study On The Alhambra}
%
%\def\autor{Ana Valdivia Garc\'{\i}a}
%
%\def\treball{Mater's degree thesis}
%
%
%\def\advisor{Salvador Garc\'{\i}a L\'{\o}pez}
%
%
%
%\vfill
%
%\def\departament{Computer Science and Artificial Intelligence}
%

%-------------------------------------------------------------%
%\maketitle

%-------------------------------------------------------------%
%%%Dedicatoria (opcional)
%-------------------------------------------------------------%
% \cleardoublepage
%\includepdf[pages={1-}]{Dedicatoria.pdf}
%\cleardoublepage
%%-------------------------------------------------------------%
%%-------------------------------------------------------------%
%%%%Prefaci (opcional)
%%-------------------------------------------------------------%
%%----------------------------------------------------------------
%% Declaraci� d'originalitat
%%------------------------------------------------------------------
%\includepdf{DeclaracionOrginalidad.pdf}
%\cleardoublepage
%%-----------------------------------------------------------------
%
%
%%\begin{comment}
%%\begin{preface}
%%\thispagestyle{empty}
%% Tot i que no �s obligat�ria la seva pres�ncia,
%%el pr�leg i el prefaci s�n dos apartats que normalment apareixen a
%%les obres. La funci� de tots dos �s transmetre opinions o reflexions
%%sobre l'obra, explicar el proc�s d'elaboraci� o argumentar-ne
%%l'oportunitat. El pr�leg el redacta una persona diferent de l'autor,
%%i el prefaci, l'autor mateix.
%%\end{preface}
%%\cleardoublepage
%%\end{comment}
%
%%---------------------------------------------------------------
%% Abstract: un resum del contingut del treball (sobre 500 paraules)
%% Ha de contenir una relaci� de paraules clau i dels codis MSC
%%(Math. Subject Classification 2000)
%% Ha d'estar tamb� en angl�s.
%%---------------------------------------------------------------
%%\includepdf[pages={1}]{pagina_en_blanc.pdf}
%
%\pagenumbering{roman} \setcounter{page}{7} \markboth{}{}
%
%\begin{abstract}
%The development of Web 2.0 has led to an important amount of content in webpage. Users are free to express their opinions about products, places and events. This project research is aimed at introducing sentiment analysis into touristic attractions. To begin with, we scrap TripAdvisor reviews from the most touristic attraction in Spain, the Alhambra. We then create two sentiment labels: the expert sentiment which is the rate of the reviewer; and the machine sentiment which is extracted from a Natural Language Processing toolkit developed in Stanford University. After that, we build classification models so as to predict polarity sentiments. Finally, we develop a subgroup discovery method so as to extract valuable information about negative reviews.
%
%\vspace{1cm}
%
%\textbf{Key words:} opinion mining, sentiment analysis, tourism, natural language processing, subgroup discovery
%\end{abstract}
%
%%\cleardoublepage
%
%%%%%%%%%%%%%%%%%%%%%%%%%%%%%%%%%%%%%%%%%%%%%%%%%%%%%%%%%%%
%%%Notation (optional)
%%%%%%%%%%%%%%%%%%%%%%%%%%%%%%%%%%%%%%%%%%%%%%%%%%%%%%%%%%%
%%\begin{notation}
%%\begin{tabular}{cl}
%%$(a,b)$ & The greatest common divisor of $a$ and $b$\\[0.3em]
%%$\parity{P}{Q}$  & Number of prime factors of $Q$, taken positively and counted with multipicity,\\[0.3em] & such that $P$ is nonresidue of them.\\[0.3em]
%
%%${\mathbb Z}$ & Nombres enters\\[0.3em]
%%${\mathbb Q}$ & Nombres racionals\\[0.3em]
%%${\mathbb R}$ & Nombres reals\\[0.3em]
%%${\mathbb C}$ & Nombres complexos
%%\end{tabular}
%%\end{notation}

%--------------------------------------------------------------
% Taula de continguts
% S'elabora autom�ticament
%-------------------------------------------------------------

\pagenumbering{roman}
\pagestyle{plain}

\setcounter{page}{10}
\includepdf[pages={1-}]{primeraspaginas}
\pagenumbering{gobble}
\tableofcontents

\clearpage

%------------------------------------------------------------
% Cos del treball
%Es poden fer servir Parts, Capitols, Seccions i Subseccions
%-------------------------------------------------------------
\pagenumbering{arabic}
\pagestyle{fancy}
\renewcommand{\thesection}{\arabic{chapter}.\arabic{section}}
\setcounter{page}{11}

\chapter{Introducción} \label{intro}
	
	\section{Contexto} \label{context}
	
	%	TESIS EUGENIO:
	%	Desarrollar aqui el comportamiento humano -> pedir / hacer opiniones
	%	Evolución histórica de la opinión -> Hasta web 2.0
	
	La principal característica del ser humano y la que le diferencia del resto de seres vivos es la racionalidad. El ser humano utiliza la razón para la toma de decisiones, contemplando los diferentes escenarios y evaluando la mejor manera de alcanzar sus objetivos. Sin embargo, el ser humano consta de una racionalidad limitada, por lo que el abanico de posibilidades que baraja en cada momento está limitado por su visión de la realidad. 
	
	Es por esto que el proceso de toma de decisiones constituye un gran reto. Para facilitar el proceso de toma de decisiones y con el fin de ampliar el abanico de consecuencias contempladas, es muy común la acción de ``pedir opinión'' a terceros.
	
	 La ayuda en la toma de decisiones no es la única función de las opiniones. Las personas también utilizamos las opiniones para expresar nuestro juicio acerca de variados temas. Cuando se produce un intercambio de opiniones ambos participantes se enriquecen con el punto de vista del contrario. 
	 
	El uso de las opiniones ha evolucionado con el ser humano a lo largo de la historia. Comenzando en los inicios del lenguaje, pasando a estar por escrito con la llegada de la escritura y disparándose con el auge de la difusión de opinión en la prensa con la invención del a imprenta.
	
	Uno de los acontecimientos más influyentes en la sociedad fue la invención de Internet en la segunda mitad del s. XX. Trajo consigo una gran fuente de información de fácil acceso. A principios del s.XXI enfatizó el cambio social que ya estaba ocurriendo un nuevo concepto de Web, la Web 2.0. Este innovador concepto ofrecía a todo usuario de ella la posibilidad de compartir todo tipo de información en Internet. Así fue como se preparó el camino para la llegada de las redes sociales, los blogs, o los foros. Plataformas donde los usuarios podían compartir todo tipo de pensamientos u opiniones sin ningún flitro. También se sumaron al carro de la Web 2.0 las empresas. Muchas de ellas incorporaron la opción de compra \textit{online} suponiendo una nueva fuente de ingresos e incluso surgieron muchas nuevas empresas que llevan a cabo toda su funcionalidad a través de Internet.
	
	%Poner aquí una frase final que no se quede como en el aire 

	
	\section{Motivación} \label{motivacion}
	%	TFG MIGUEL:
	%	Conectando con lo anterior -> hay muchísimas opiniones en internet
	%	Aparición de grandes empresas
	%	Las empresas quieren/necesitan conocer la opinión/éxito/fracaso de sus productos 
	%	para los procesos de marketing y para posibles mejoras

	En el contexto de la Sección \ref{context}, tanto el auge de las redes sociales como la incorporación de las empresas al negocio \textit{online} no trajeron consigo solo nuevas fuentes de beneficios. La Web 2.0 supuso una nueva (y enorme) fuente de información. Esta generación exponencial de datos de naturaleza desestructurada en Internet propició el nacimiento de lo que hoy conocemos como \textit{Big Data}.
	
	\begin{figure}[h!]
		\centering
		\includegraphics[width=0.7\linewidth]{imagenes/grafica-intro}
		\caption{Gráfica del crecimiento del \textit{Big Data} en algunos países y en el mundo.}
		\label{fig:grafica-intro}
	\end{figure}
	 
	 Para hacernos una idea del impacto del nacimiento de la Web 2.0, observamos la Figura \ref{fig:grafica-intro} en la que se representa el crecimiento de \textit{Big Data} en algunos países desarrollados y en el mundo en general. Notamos que en el año 1998 se produce un aumento en el ritmo de crecimiento, coincidiendo con el nacimiento de la Web 2.0. El siguiente cambio de crecimiento más brusco lo encontramos en el año 2006, año de mayor impacto de las redes sociales. 
	 
	 %Quizás nombrar en algún momento microblogging (Twitter) y esop
	 
	 Como ya veníamos advirtiendo, está en la naturaleza del ser humano dar su opinión sobre los temas que le rodean. Esta caracterización del ser humano junto con la cantidad de datos depositados por usuarios en los diferentes portales de Internet nos da una idea de la cantidad de opiniones disponibles. Esta información será de gran utilidad para las empresas que quieran conocer la opinión de los usuarios sobre un determinado producto o, simplemente, para conocer la opinión de la población sobre un determinado tema.  Sin embargo, la inmensurable cantidad de información disponible hace que contratar a personal especializado para que se dedique a estudiar las opiniones de un determinado producto o tema de actualidad sería inviable tanto desde el punto de vista económico como desde el temporal.  Debido a la necesidad de monitorizar este procedimiento surge el concepto de Análisis de Sentimientos o Minería de Opinión, del que hablaremos en los siguientes capítulos.
	 
	 Por su estructura de \textit{microblogging} (red social con un número de caracteres limitados para resaltar la opinión), Twitter es un perfecto generador de opiniones para analizar mediante Análisis de Sentimientos. Ya se han llevado a cabo estudios sobre esta plataforma obteniendo muy buenos resultados. Por ejemplo, en las últimas elecciones presidenciales de EEUU se realizó un estudio sobre la opinión de los usuarios de Twitter sobre ambos candidatos. Para este estudio se utilizaron los tweets publicados hasta 43 días antes de las elecciones comenzando el primer día de candidatura. Utilizando expertos para el etiquetado de los datos como positivos o negativos referentes a cada uno de los candidatos y entrenando con un algoritmo de Naïve Bayes consiguieron un 94\% de correlación.
	 
	 El término de Análisis de Sentimientos es un término relativamente reciente. Aunque ya haya tenido éxito en algunos ámbitos, es un campo aún sin explotar pues presenta dificultades que aún no han sido solventadas.  Entre estas dificultades se encuentran la subjetividad de las opiniones, las particularidades del lenguaje, los contextos, los sarcasmos ... entre una interminable lista.
	 
	 
	 \begin{figure}[h!]
	 	\centering
	 	\includegraphics[width=1\linewidth]{imagenes/interes}
	 	\caption{Gráfica que representa el interés del análisis de sentimientos en todo el mundo según Google Trends.}
	 	\label{fig:interes}
	 \end{figure}
	 
	 Esto hace que se presente como una campo de estudio muy llamativo, con muchas mejoras por hacer y con un futuro muy prometedor. Cada día más gente centra su interés en esta materia (Figura \ref{fig:interes}). De ahí que a lo largo de esta memoria nos dediquemos al estudio de una parte de este amplio campo.
	 

	Como ya podemos imaginar, el término Análisis de Sentimientos abarca un gran conjunto de tareas. El trabajo se va a centrar en una de esas tareas en concreto: la extracción de características. 

	Para explicar en qué consiste y su importancia, trabajaremos sobre un ejemplo concreto. Imaginemos que queremos analizar la siguiente opinión:
	
	\begin{center}
		\begin{minipage}{0.9\linewidth}
			\vspace{5pt}%margen superior de minipage
			{\small
				[1] \textit{El ordenador tiene muy buen procesador, sin embargo la pantalla tiene poca resolución y el precio es muy elevado.}
			}
			\vspace{5pt}%margen inferior de la minipage
		\end{minipage}
	\end{center}

	Si tuviéramos que establecer una polaridad a dichar frase, sería una tarea complicada incluso para una persona. En la frase se nombran tres componentes de un mismo ordenador y se da un opinión sobre cada una de ellas. Así, \textit{buen procesador} tendría connotaciones positivas mientras que \textit{la pantalla tiene poca resolución} y \textit{el precio es muy elevado} negativas. La polaridad general del producto dependerá de la importancia que se le dé a cada una de las partes. Como no podemos saber a priori un orden de prioridad establecido, sería muy útil poder dar una polaridad a cada una de las componentes.
	
	Ahora bien, para poder realizar este proceso de forma automatizada nos encontraríamos con dos tareas: en primer lugar la identificación de la característica y, en segundo lugar, la detección de la polaridad de esta. En este trabajo nos centraremos en la primera de estas tareas. 

	\section{Objetivos} \label{objetivos}
	%	Hablar de los objetivos del trabajo en concreto
	%	Capítulos en los que se divide el trabajo.
	
	
	
		

	

\chapter{Análisis de sentimientos}

\section{Concepto de Análisis de Sentimientos} \label{conceptsentiment}




\chapter{Fundamentos matemáticos}

\textcolor{red}{Concepto de derivada y definiciones alternativas.}

\textcolor{red}{Uso de la derivada en minimización de funciones.}

\textcolor{red}{Gradiente para una función en Rn}

\textcolor{red}{Composición de funciones.}

\textcolor{red}{Grafos dirigidos.}

\textcolor{red}{ÁLGEBRA TENSORIAL -- Optimización de algoritmos deeplearning (tensorflow)}

\textcolor{red}{PROBABILIDAD y TEORIA DE LA INFORMACION}
\chapter{Aprendizaje Automático}

	En este capítulo desarrollaremos los principios básicos de aprendizaje automático para introducir el \textit{Deep Learning} en capítulos posteriores. Motivaremos su uso con algunos ejemplos y nos adentraremos en el problema de clasificación pues es el que desarrollaremos durante el resto de la memoria. 
	
	Finalmente hablaremos del gradiente descendente, algoritmo de optimización en el cual se inspirará el Deep Learning.
	
	
\section{Concepto de Aprendizaje Automático}\label{introap}

	Un algoritmo de \textbf{Aprendizaje Automático} es un algoritmo que es capaz de aprender información a partir de ciertos datos.  Pero, ¿qué entendemos por aprender? Según \cite{mitchell}, ``Un programa de ordenador se dice que aprende de una experiencia \textit{E} con respecto a un conjunto de tareas \textit{T} y medida de rendimiento \textit{P}, si su rendimiento como tarea en \textit{T}, medido con \textit{P}, mejora tras conocer la experiencia \textit{E}''. Como podemos imaginar, estas tareas y experiencias son muy diversas, de donde surgen gran variedad de técnicas de aprendizaje con estas características.
	
	Desde  el punto de vista de las tareas \textit{T}, el aprendizaje automático es interesante pues nos permite tratar con tareas que son muy difíciles de resolver con programas escritos y diseñados por humanos, ya sea por la complejidad de la solución o por el tamaño de la tarea.
	
	Han sido muchas las aplicaciones de esta clase de algoritmos. En la literatura podemos encontrar varios ejemplos:
	
	\begin{itemize}
		\item Clasificación
		\item Clasificación con valores perdidos
		\item Regresión
		\item Transcripción
		\item Detección de anomalías
		\item Predicción de valores pedidos
		\item Estimación de densidad
	\end{itemize}
	
	Con respecto a la medida de rendimiento, \textit{P}, debemos diseñar una medida cuantitativa. La medida por excelencia para casos de clasificación y transcripción es la exactitud o \textit{accuracy} de la solución, definida como la proporción de muestras para las que el modelo ha producido una salida correcta. También puede ser útil en algunos casos la tasa de error o \textit{error rate}, que mide la proporción de muestras para las que el modelo ha producido uan salida incorrecta. 
	
	Para aquellas tareas como la estimación de densidad, en las que la salida es continua no tiene sentido utilizar las medidas anteriormente comentadas. En estos casos se utilizarán medidas que proporcionen resultados continuos.
	
	\subsection{Tipos de aprendizaje automático.}
	
	En función de la experiencia \textit{E}, podemos clasificar los algoritmos de aprendizaje automático en dos grandes grupos: algoritmos supervisados y no supervisados.
	
	
	\begin{itemize}
		\item \textbf{Aprendizaje Automático no supervisado}: hablamos de este tipo de aprendizaje cuando los datos con los que trabajamos se conforman de un conjunto de datos que representan diferentes características de cada una de las muestras. De este modo, el objetivo suele ser encontrar propiedades importantes para la estructura del conjunto de datos. Un ejemplo clásico de problema de este tipo es el \textit{clustering}, que consiste en la agrupación de los datos en diferentes subconjutnos que contengan características similares.
		
		\item \textbf{Aprendizaje Automático supervisado}: se obtiene información de un conjunto con características pero donde cada muestra tiene asociada una etiqueta. Así, el objetivo de este tipo de problemas es encontrar una relación entre la etiqueta y las características, pudiendo etiquetar muestras futuras. Un ejemplo de este tipo de problema es la clasificación, en el que entraremos en más detalle a continuación.
	\end{itemize}

	Prácticamente todo el Deep Learning está motivado por el \textbf{Gradiente Descendente Estocástico}. Este algoritmo es una extensión del Gradiente Descendente. Por tanto, a continuación haremos una breve introducción a estos dos conceptos.
	
	
	\section{Problema de clasificación}
	
	\textcolor{red}{Encontrar referencia formal donde pueda yo explicar esto}
	
	\section{Optimización basada en Gradiente Descendente}
	
	La mayoría de los algoritmos de Deep Learning involucran algún tipo de optimización. La función que queremos optimizar $f(x)$ se denomina \textbf{función objetivo}. 
	
	Suponemos que tenemos una función $y = f(x)$ con $x,y \in \mathbb{R}$. La derivada de esta función, $f'(x)$ nos el crecimiento/decrecimiento de la misma. Es decir, nos indica cómo escalar un pequeño cambio en la entrada, para obtener el correspondiente cambio en la salida. Esto es, usando la definición clásica de derivada con valores muy pequeños de $\epsilon$:
	
	$$
		f'(x) = \lim_{\epsilon \rightarrow 0} \frac{f(x+ \epsilon) - f(x)}{\epsilon} \Rightarrow f(x+ \epsilon) \approx f(x) + \epsilon f'(x)
	$$
	
	De este modo podríamos minimizar la funciñon $f$ pues sabríamos cómo variar $x$ para realizar una mejora en $y$. Por ejemplo, si $f(x - \epsilon$sign$(f'(x))) < f(x)$ para $\epsilon > 0$ arbitrario, entonces podemos reducir $f(x)$ sin más que mover $x$  en el sentido contrario al signo de la derivada. Esta técnica es lo que se conoce como gradiente descendente para una variable. 
	
	En el caso de varias variables, el procedimiento es muy similar. Consideraremos el gradiente de la función $f: \mathbb{R^n} \rightarrow \mathbb{R}$ en $x$, $\nabla_x f(x)$ y nos movemos una cantidad $\epsilon > 0$ en misma dirección y sentido contrario de la forma:
	
	$$
		x' = x - \epsilon \nabla_x f(x)	
	$$ 

	El algoritmo multivariante termina cuando $\nabla_x f(x)$ es 0 o muy cercano según una tolerancia prefijada.
	
	Destacar que la velocidad con la que avance el método variará en función del valor tomado para $\epsilon$ en ambos casos. Así, si elegimos un valor de $\epsilon$ muy pequeño, el método puede avanzar de forma demasiado lenta mientras que si elegimos un valor muy alto puede producirse un efecto \textit{zigzag} alrededor de un óptimo local
	
	\textcolor{red}{Ejemplos?}

	\subsection{Gradiente Descendente Estocástico}
	
	Un problema del Aprendizaje Automático radica en que para obtener suficiente generalización, el conjunto de datos sobre el que entrenamos los datos debe ser lo suficientemente grande. Esto conlleva una carga muy elevada de cómputo, pues normalmente la función de coste se calcula como la suma del valor en cada uno de los valores de la muestra.
	
	El Gradiente Descendente Estocástico (GDS) se basa en la idea de subsanar este problema utilizando solo un conjunto pequeño de muestras, llamado \textbf{minibatch} $\mathbb{B} = \{x^{(1)}, ..., x^{(m)}\}$, donde $m$ representa una pequeña muestra en función con el total del conjunto de datos.
	
	De esta forma, la estimación del gradiente sería
	
	$$ 
		g = \frac{1}{m} \nabla_{\theta} \sum_{i = 1}^{m} L(x^{(i)}, y^{(i)}, \theta)
	$$
	
	donde $x \in \mathbb{B}$. A continuación, mediante el algoritmo del Gradiente Estocástico descendente se estimaría el gradiente de la siguiente forma
	
	$$ 
		\theta \leftarrow \theta - \epsilon g
	$$
	
	donde $\epsilon$ es la tasa de aprendizaje.
	
	
	El Deep Learning se ve motivado porque el Aprendizaje Automático no había conseguido resolver los problemas centrales de la Inteligencia Artificial, como por ejemplo el reconocimiento de imágenes o procesamiento natural del lenguaje.
	



\chapter{Deep Learning}

\section{Redes Neuronales Prealimentadas}

Las \textbf{redes neuronales} son el ejemplo más típico de modelos de Deep Learning. El objetivo de esta clase de algoritmos es aproximar una función $f^*$. Por ejemplo, para un clasificador $y = f^*(x)$ que asocia a cada entrada $x$ una categoría $y$. Así, una red neuronal prealimentada define una asociación $y = f(x; \theta)$ y aprende los valores de los parámetros $\theta$ que ofrecen la mejor aproximación de la función.

Esta clase de modelos se llaman \textbf{prealimentados} porque la información fluye desde la función siendo evaluados desde $x$, mediante los cálculos intermedios usados para definir $f$ y finalmente hasta la salida $y$. 

Las redes neuronales son llamadas \textbf{redes} pues se definen mediante la composición de diferentes funciones. Este modelo de sucesivas composiciones se puede ver como un grafo dirigido acícilo. 

El número de capas que tenga la red definen la \textbf{profundidad} del modelo. Las capas se van nombrando de forma sucesiva siendo la primera la primera capa o \textit{capa de entrada}, \textit{segunda capa}, y así sucesivamente hasta la capa final denominada \textit{capa de salida}. El objetivo del modelo es conducir la función de entrada $f(x)$ para que se ajuste a $f^*(x)$. 

\section{Redes Neuronales Convolucionales}



\section{Redes Neuronales Recurrentes}

\subsection{Long-Short Term Memory}

%\cleardoublepage
\let\cleardoublepage\clearpage
%\appendix{Fora del cos} Els ap�ndixs poden contenir demostracions
%especialment t�cniques, prgrames d'ordinador, descripci�
%d'algorismes, etc.


%\printindex

%\input{bibliografia}

%\cleardoublepage \sloppy
%\begin{raggedright}
%\input{exemple_tfm.ind}
%\end{raggedright}



%--------------------------------------------------------------


%--------------------------------------------------------------


%
\begin{thebibliography}{99}
	%\bibitem[Daniel Bestard Delgado]{intro1} ¿Cómo se explica el crecimiento del BIG DATA en la última década? ¿Qué retos nos ha planteado esta ciencia?
	%\bibitem[Liu, 2012]{liu} \hspace{-.22cm}  Sentiment analysis and opinion mining. Synthesis lectures on human language technologies. Liu, B. (2012).
	\bibitem[Mitchell, 1997]{mitchell} \hspace{-.22cm} \textit{Machine Learning.} McGraw-Hill, New York.
	
	\bibitem[Goodfellow, 2016]{goofellow} \hspace{-.22cm} \textit{Deep Learning.} Gooldfellow, Ian. \textbar  Bengio, Yosuha. \textbar	  Courville, Aaron. 

\end{thebibliography}
\end{document}
%--------------------------------------------------------------%
%--------------------------------------------------------------%
